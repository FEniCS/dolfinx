\chapter{Partial differential equations}
\label{sec:pde}
\index{partial differential equations}

\section{Boundary value problems}

As a prototype of a boundary value problem in $\Bbb{R}^d$ we consider the 
scalar Poisson equation with homogeneous Dirichlet boundary conditions 
\begin{eqnarray}
-\Delta u(x)&=&f(x) \quad x\in \Omega \subset \Bbb{R}^d \label{pde:poisson:strong} \\
u(x)&=&0 \quad x\in \partial \Omega. \nonumber  
\end{eqnarray}

\section{Variational formulation}

A variational formulation of (\ref{pde:poisson:strong}) take the form: 
find $u\in V$ such that  
\begin{equation}\label{pde:poisson:weak}
a(u,v)=L(v) \quad \forall v\in V, 
\end{equation}
where $a(\cdot,\cdot):V\times V\rightarrow \Bbb{R}$ is a bilinear form 
on $V$ defined by 
\begin{equation}
a(u,v)=\int_{\Omega} \nabla u \cdot \nabla v ~dx 
=\int_{\Omega} \frac{\partial u}{\partial x_i} \frac{\partial v}{\partial x_i} ~ dx,  
\end{equation}
where we employ tensor notation so that the double index $i$ means summation from $i=1,...,d$, 
and $L(\cdot):V\rightarrow \Bbb{R}$ is a linear form on $V$ defined by 
\begin{equation}
L(v)=\int_{\Omega} f v ~dx.  
\end{equation}
$V=H^1_0(\Omega)$ is the standard Sobolev space of square integrable 
functions with also their first derivatives square integrable (in the Lebesgue sense), 
with the functions being zero on the boundary (in the sense of traces).   

The Finite Element Method FEM for (\ref{pde:poisson:weak}) is now: 
find $U\in V_h$ such that  
\begin{equation}\label{pde:poisson:fem}
a(U,v)=L(v) \quad \forall v\in V_h, 
\end{equation}
where $V_h\subset V$ is a finite dimensional subspace of dimension $N$. 
The finite element space $V_h$ is characterized by the set of basis 
functions $\{\varphi_i\}_{i=1}^N$, and thus the FEM method 
(\ref{pde:poisson:fem}) is specified by the variational form and 
the basis functions of $V_h$. 

\section{Compiling the variational form with FFC}

In \dolfin{} a PDE is defined in variational form using tensor notation 
in a \texttt{.form} file, which is compiled using FFC. 

In FFC, (\ref{pde:poisson:fem}) is defined by:  
\begin{code}
element = FiniteElement("Lagrange", "tetrahedron", 1)

v = BasisFunction(element)
u = BasisFunction(element)
f = Function(element)

a = v.dx(i)*u.dx(i)*dx
L = v*f*dx
\end{code}

Compiling the file with 
\begin{code}
# ffc Poisson.form
\end{code}
generate a file \texttt{Poisson.h} containing the classes 
\texttt{BilinearForm} and \texttt{LinearForm}, and 
classes for the finite element used in the forms. 

\section{Assemble matrices and vectors}

The class \texttt{FEM} automates the assembly algorithm, constructing a linear 
system of equations from a given partial differential equation, 
given in the form of a variational problem (\ref{pde:poisson:weak}), 
with a bilinear form $a(\cdot,\cdot)$ and a linear form $L(\cdot)$. 

The classes \texttt{BilinearForm} and \texttt{LinearForm} are automatically 
generated by FFC, and to assemble the corresponding matrix and vector for 
the Poisson problem with source term $f$, we write:  
\begin{code}
Poisson::BilinearForm a;
Poisson::LinearForm L(f);

Mesh mesh; 
Mat A;
Vec b;

FEM::assemble(a,L,A,b,mesh);
\end{code}

In the \texttt{assemble} function the element matrices and vectors are 
computed by calling the function \texttt{eval} in the classes 
\texttt{Bilinearform} and \texttt{Linearform}. 
The \texttt{eval} function at a certain element in the assebly algorithm 
takes as argument an \texttt{AffineMap} object, 
describing the mapping from the reference element to the actual element, 
by computing the jacobian $J$ of the mapping (also $J^{-1}$ and $det(J)$ 
are computed).  

\section{Boundary conditions}

\section{Finite elements}

Finite Element by Ciarlet, FIAT, FFC  

\section{Computation of Element matrices and vectors} 

divide element matrix into geometry tensor and integration 
over reference element FErari

precomputation of integrals, quadrature, tensorrepresentation factored out, FFC 

\section{Initial value problems}

semidiscretization, space-time FEM, 

% Insert note that constants are passed into forms as references
