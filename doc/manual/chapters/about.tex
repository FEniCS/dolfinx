% This chapter is common to the DOLFIN and FFC manuals.

\addcontentsline{toc}{chapter}{About this manual}
\chapter*{About this manual}

This manual is currently being written. As a consequence, some sections
may be incomplete or inaccurate.

%------------------------------------------------------------------------------
\section*{Intended audience}

This manual is written both for the beginning and the advanced user.
There is also some useful information for developers. More advanced topics
are treated at the end of the manual or in the appendix.

%------------------------------------------------------------------------------
\section*{Typographic conventions}
\index{typographic conventions}

\begin{itemize}
\item
  Code is written in monospace (typewriter) \texttt{like this}.
\item
  Commands that should be entered in a Unix shell
  are displayed as follows:
  \begin{code}
    # ./configure
    # make
  \end{code}
  Commands are written in the dialect of the \texttt{bash} shell. For
  other shells, such as \texttt{tcsh}, appropriate translations may be
  needed.
\end{itemize}

%------------------------------------------------------------------------------
\section*{Enumeration and list indices}
\index{enumeration}
\index{indices}

Throughout this manual, elements $x_i$ of sets $\{x_i\}$ of size $n$
are enumerated from $i = 0$ to $i = n-1$. Derivatives in $\R^n$ are
enumerated similarly:
$\frac{\partial}{\partial x_0}, \frac{\partial}{\partial x_1},
 \ldots, \frac{\partial}{\partial x_{n-1}}$.

%------------------------------------------------------------------------------
\section*{Contact}
\index{contact}

Comments, corrections and contributions to this manual are most welcome
and should be sent to
\begin{code}
  \packagett{}-dev@fenics.org    
\end{code}
