\chapter{Contributing code}
\index{contributing}

If you have created a new module, fixed a bug somewhere, or have made
a small change which you want to contribute to \dolfin{}, then the
best way to do so is to send us your contribution in the form of a
patch. A patch is a file which describes how to transform a file or
directory structure into another. The patch is built by comparing a
version which both parties have against the modified version which
only you have.

%------------------------------------------------------------------------------
\section{Creating a patch}
\index{diff}
\index{patch}

The tool used to create a patch is called \texttt{diff} and the tool
used to apply the patch is called \texttt{patch}. These tools are free
software and are standard on most Unix systems.

Here's an example of how it works. Start from the latest release of
\dolfin{}, which we here assume is release 0.5.0. You then have a
directory structure under \texttt{dolfin-0.5.0} where you have made
modifications to some files which you think could be useful to
other users.

\begin{enumerate}
\item
  Clean up your modified directory structure to remove temporary and binary
  files which will be rebuilt anyway:
  \begin{verbatim}
    # make clean
  \end{verbatim}
\item
  From the parent directory, rename the \dolfin{} directory to something else:
  \begin{verbatim}
    # mv dolfin-0.5.0 dolfin-0.5.0-mod
  \end{verbatim}
\item
  Unpack the version of \dolfin{} that you started from:
  \begin{verbatim}
    # tar zxfv dolfin-0.5.0.tar.gz
  \end{verbatim}
\item
  You should now have two \dolfin{} directory structures in your current directory:
  \begin{verbatim}
    # ls
    dolfin-0.5.0
    dolfin-0.5.0-mod
  \end{verbatim}
\item
  Now use the \texttt{diff} tool to create the patch:
  \begin{verbatim}
    # diff -u --new-file --recursive dolfin-0.5.0
      dolfin-0.5.0-mod > dolfin-<identifier>-<date>.patch
  \end{verbatim}
  written as one line, where \texttt{<identifier>} is a keyword that
  can be used to identify the patch as coming from you (your username,
  last name, first name, a nickname etc) and \texttt{<date>} is
  today's date in the format \texttt{yyyy-mm-dd}.
\item
  The patch now exists as \texttt{dolfin-<identifier>-<date>.patch}
  and can be distributed to other people who already have
  \texttt{dolfin-0.5.0} to easily create your modified version. If the
  patch is large, compressing it with for example \texttt{gzip} is
  advisable:
  \begin{verbatim}
    # gzip dolfin-<identifier>-<date>.patch
  \end{verbatim}
\end{enumerate}

%------------------------------------------------------------------------------
\section{License agreement}
\index{license}

By contributing a patch to \dolfin{}, you agree to license your
contributed code under the GNU General Public License (a condition
also built into the GPL license of the code you have modified). Before
creating the patch, please update the author and date information of
the file(s) you have modified according to the following example:

\begin{verbatim}
  // Copyright (C) 2004-2005 Johan Hoffman and Anders Logg.
  // Licensed under the GNU GPL Version 2.
  //
  // Modified by Johan Jansson 2005.
  // Modified by Garth N. Wells 2005.
  //
  // First added:  2004-06-22
  // Last changed: 2005-09-01
\end{verbatim}

As a rule of thumb, the original author of a file holds the copyright.

%------------------------------------------------------------------------------
\section{Sending patches}
\index{patch}

Patch files should be sent to the \dolfin{} mailing list at the address
\begin{verbatim}
  dolfin-dev@fenics.org
\end{verbatim}
Include a short description of what your patch accomplishes. Small
patches have a better chance of being accepted, so if you are making a
major contribution, please consider breaking your changes up into
several small self-contained patches if possible.

%------------------------------------------------------------------------------
\section{Applying a patch (maintainers)}
\index{patch}

Let's say that a patch has been built relative to \dolfin{} release 0.5.0.
The following description then shows how to apply the patch to a clean
version of release 0.5.0.

\begin{enumerate}
\item
  Unpack the version of \dolfin{} which the patch is built relative to:
  \begin{verbatim}
    # tar zxfv dolfin-0.5.0.tar.gz
  \end{verbatim}
\item
  Check that you have the patch \texttt{dolfin-<identifier>-<date>.patch} and the \dolfin{}
  directory structure in the current directory:
  \begin{verbatim}
    # ls
    dolfin-0.5.0
    dolfin-<identifier>-<date>.patch
  \end{verbatim}
  Unpack the patch file using \texttt{gunzip} if necessary.
\item
  Enter the \dolfin{} directory structure:
  \begin{verbatim}
    # cd dolfin-0.5.0
  \end{verbatim}
\item
  Apply the patch:
  \begin{verbatim}
    # patch -p1 < ../dolfin-<identifier>-<date>.patch
  \end{verbatim}
  The option \texttt{-p1} strips the leading directory from the filename
  references in the patch, to match the fact that we are applying the
  patch from inside the directory. Another useful option to
  \texttt{patch} is \texttt{--dry-run} which can be used to test the
  patch without actually applying it.
\item
  The modified version now exists as \texttt{dolfin-0.5.0}.
\end{enumerate}
